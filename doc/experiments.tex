\section{Experiments} \label{sec:exp}

This section describes the Travelling Salesperson Problem (TSP) and the Google
Compute Engine's virtual machines and project settings used to evaluate the
efficacy of the proposed methodology and protocol.  The performances of
the autotuner for the TSP were measured in 8 different
experimental settings. Each tuning run lasted 15 minutes, used 2 or 4
virtual machines, and was repeated 4 times.  We also varied the number of
result requests that each virtual machine in a tuning run processed, namely 1,
4, 8 or 16 requests per machine.

\subsection{Using the Google Compute Engine}

All virtual machines used in the experiments had a single vCPU and 3.75GB of
RAM (Google Compute Engine machine type \texttt{\footnotesize n1-standard-1}).
All experiments were performed with machines from the \texttt{\footnotesize
us-central1-f} zone. We built a virtual machine image with the latest stable
Debian distribution and all dependencies installed, speeding up the virtual
machines' initialization time. The local machine was an 8-core machine

\subsection{Travelling Salesperson Problem}

The instances of the TSP used in the
experiments in this paper were obtained from TSPLIB~\cite{reinelt1991tsplib}.
A TSP solver was implemented as an OpenTuner application. The search space was
defined by all the possible permutations, or tours, of cities where the first
and last cities are the same. We used two instances, of size 532 and 85900.
