\section{Introduction} \label{sec:intro}

The program autotuning problem fits in the framework of the Algorithm Selection
Problem, introduced by Rice in 1976~\cite{rice1976algorithm}. The objective of
an autotuner is to select the best algorithm, or algorithm configuration, for
each instance of a problem.  Algorithms or configurations are selected
according to performance metrics such as the time to solve the problem
instance, the accuracy of the solution and the energy consumed.  The set of all
possible algorithms and configurations that solve a problem defines a
\emph{search space}. Various optimization techniques search this space, guided
by the performance metrics, for the algorithm or configuration that best solve
the problem.

There are specialized autotuners for domains such as matrix
multiplication~\cite{bilmes1997phipac}, dense~\cite{whaley1998atlas} or
sparse~\cite{vuduc2005oski} matrix linear algebra, and parallel
programming~\cite{jordan2012multi}. Other autotuning frameworks provide more
general tools for the representation and search of program configurations,
enabling the implementation of autotuners for different problem
domains~\cite{ansel2014opentuner,hutter2009paramils}.

The OpenTuner framework~\cite{ansel2014opentuner} provides tools for the
implementation of autotuners for various problem domains. It implements
different search techniques that explore the same search space for program
optimizations. 
Running and measuring program execution time --- that is, the empirical 
exploration of the search space --- is done sequentially.
The framework provides support for parallel compilation.

The contributions of this paper are a new methodology and a protocol for the
distributed execution of autotuners using cloud computing resources.  The
methodology and protocol are implemented as extensions to the OpenTuner
framework distributing and parallelizing the exploration of optimization spaces
by combining results obtained from virtual machines.  A local machine (LM) runs
the main OpenTuner application. Several virtual machines (VM) run measurement
modules and provide results when requested. This distributed approach performs
a more cost efficient exploration of the search space in comparison with the
sequential execution.

The interactions between the local and virtual machines follows the
client-server model. The local machine runs a measurement client that requests
results from various measurement servers running in virtual machines hosted at
a public cloud. The experiments in this paper used the Google Compute Engine
(GCE).  We compare the performance of our extension with the unmodified
framework in a benchmark of applications, identifying the problem domains that
benefit from this cloud-based approach.

The rest of the paper is organized as follows.
Section~\ref{sec:related} discusses related work.
Section~\ref{sec:ot} describes the architecture of the OpenTuner framework.
Section~\ref{sec:ext} presents the proposed methodology, the architecture of 
the measurement driver extension, the GCE interface and the application 
protocol.
Section~\ref{sec:norm} discusses the result normalization strategies.
Section~\ref{sec:exp} describes the experiments performed and the
applications used in the benchmark.
Section~\ref{sec:results} discusses the results obtained.
Section~\ref{sec:conclusion} concludes the paper.
